% !TEX encoding = UTF-8
% !TEX program = pdflatex
% !TEX root = ../guidaconsole.tex

\chapter{Argomenti avanzati}
\label{chapAvanz}

In questo capitolo presenterò alcune procedure avanzate che sfruttano la riga di
comando per compiere operazioni altrimenti noiose, ripetitive o troppo lunghe da
svolgere con altri strumenti. Si tratta di una collezione di casi pratici
risolti con l'uso avanzato della riga di comando che si dimostra essere un
sistema molto potente.

% TODO: argomenti ancora da scrivere ed inserire
%\section{Creazione di cartelle ramificate}
%\section{Aprire il terminale da una cartella}
%\section{Redirezione dei dati}
%\section{Operare con i diritti di root}

\section{Espressioni regolari}

Di solito le chiavi delle \emph{espressioni regolari} sono piuttosto
indecifrabili\dots{} devono infatti rispondere a formati e caratteri di
controllo poco amichevoli ma hanno il vantaggio di non richiedere
programmazione.

Negli esempi, useremo il formato RE\footnote{RE è l'acronimo di \emph{regular
expression}.} previsto dal linguaggio Perl, poiché il compilatore \texttt{perl}
è disponibile anche per Windows.

Per disporre di Perl, linguaggio nato appositamente anche per manipolare file di
testo, gli utenti Windows possono o eseguire una installazione dal sito
\url{www.perl.org} oppure, se dispongono di \TeX{} Live, possono tentare in via
sperimentale di utilizzare il compilatore \texttt{perl} minimale già compreso
nella distribuzione inserendo nel \texttt{PATH} utente il percorso
\texttt{\$tlhome\textbackslash\$year\textbackslash tlpkg\textbackslash
tlperl\textbackslash bin} dove \texttt{\$tlhome} va sostituito con la cartella
contenente \TeX{} Live, solitamente \texttt{C:\textbackslash texlive} ed
\texttt{\$year} va sostituito con il numero di versione che corrisponde all'anno
di uscita.

La chiave RE per sostituire un testo con un altro è '\verb=s/<old>/<new>/='
composta dal carattere \texttt{s} che sta per \emph{singola linea} ed il
carattere slash \texttt{/} che fa da separatore. In coda alla chiave possono
essere aggiunti \emph{modificatori} per esempio \texttt{i} che significa che la
ricerca dell'occorrenza non deve tener conto di maiuscolo/minuscolo
(\emph{ignore}), e \texttt{g} che indica che devono essere trovate tutte le
occorrenze nella riga (\emph{global}) e non solo la prima. Per maggiori
approfondimenti visitare la pagina web
\href{http://perldoc.perl.org/perlrequick.html}{Perl RE Quick Guide} della
documentazione Perl.


\subsection{Rinominare file}

Diversamente dall'implementazione in Windows, in Linux il comando
\texttt{rename} accetta una chiave nel formato
\href{http://perldoc.perl.org/perlrequick.html}{Perl Regular Expression},
rendendolo molto flessibile e potente. In Windows rimane la possibilità di
utilizzare un comando scritto in Perl che si comporti come il \texttt{rename} di
Linux anche se questa soluzione richiede un po' di lavoro in più.

Ammettiamo che in una sotto cartella \texttt{image} i nomi delle immagini
contengano degli spazi. Questo non permette il regolare funzionamento delle
macro \LaTeX{} per l'inclusione dei file. Con un unico comando decidiamo di
sostituire tutti gli spazi con il carattere trattino \texttt{-}. Formiamo la
chiave RE che inizia con \texttt{s} con il carattere da sostituire (lo spazio),
il nuovo carattere (il trattino \texttt{-}), ed aggiungiamo il modificatore
\texttt{g} (global) per sostituire \emph{tutte} le occorrenze:
\begin{verbatim}
$ rename 's/ /-/g' image/*
\end{verbatim}

Ancora per evitare problemi con le immagini in un progetto \LaTeX, vogliamo
rendere minuscole le lettere che compongono i nomi dei file compreso quelli
dell'estensione. Il modello RE indicherà stavolta una trasliterazione
specificando non un unico carattere ma il set di caratteri \texttt{A-Z} e quelli
corrispondenti da sostituire con il set \texttt{a-z}. La manipolazione in
corrispondenza di posizione richiede le lettere iniziali \texttt{tr},
\textbf{tr}ansliterate (che possono essere scritte con il sinonimo \texttt{y}):
\begin{verbatim}
$ rename 'y/A-Z/a-z/' image/*
\end{verbatim}

La trasformazione dei caratteri spazio in trattini del primo esempio, può essere
ottenuta anche con il modello RE di traslitterazione \texttt{'y/ /-/'}.

Potremo voler rinominare i file premettendo ai nomi un prefisso, per esempio
\texttt{123}, con il modello \verb='s/^/123/'= oppure, viceversa, potremo voler
eliminarlo con il modello \verb='s/^\w{3}//'=, oppure ancora eliminare
escludendo l'estensione, l'ultimo carattere del nome, qualsiasi esso sia, con il
modello \verb='s/\w\./\./'=.


\subsection{Correzione massiva di sorgenti}

Si tratta del problema della sostituzione di occorrenze in un testo che possiamo
risolvere avvalendoci dei comandi di cerca/sostituisci presenti negli editor ma
se sono coinvolti molti file è più conveniente l'uso del terminale. La sintassi
generale del comando Perl da riga di comando è la seguente:
\begin{verbatim}
$ perl -p -i -e 's/<RegExpr>/<nuovo>/modificatori' <files>
\end{verbatim}

Esaminiamo le tre opzioni iniziali: \texttt{-p} indica che dovranno essere
esaminate tutte le righe di testo nei file, \texttt{-i} indica che saranno
modificati direttamente i file originali (opzione \emph{inline}) e \texttt{-e}
indica che dovrà essere eseguito il codice specificato di seguito (opzione
\emph{execute}).

Abbiamo poi l'argomento che specifica il pattern di sostituzione con campi
separati da uno slash, nel formato già incontrato per il comando
\texttt{rename}.

Nell'eseguire questi comandi si deve prestare attenzione verificando la corretta
esecuzione. \`E consigliabile fare una copia di backup dei file perché potrebbe
presentarsi la necessità di ripristinare i file originali, e ragionare sulla
carta elaborando tutti i casi possibili.


\subsubsection{Sostituire una parola}

Il nostro revisore ci ha segnalato gentilmente che nella tesi anziché scrivere
``perché'' abbiamo invece utilizzato l'accento sbagliato scrivendo ``perchè''.
Dopo aver compreso l'errore\footnote{Leggendo per esempio l'articolo ``Norme
tipografiche per l’italiano in \LaTeX'' di Gustavo Cevolani disponibile
all'indirizzo web sul
\href{http://www.guitex.org/home/images/ArsTeXnica/AT001/norme\%20tipografiche\%20per\%20litaliano\%20in\%20latex.pdf}{sito
del \GuIT*}.} eseguiamo una copia di sicurezza dei sorgenti e lanciamo il
comando seguente una volta impostata la working directory a quella che contiene
la cartella \texttt{chapter} con i file da correggere:
\begin{verbatim}
$ perl -p -i -e 's/perchè/perché/g' chapter/*.tex
\end{verbatim}

Nel testo potrebbero esserci dei ``Perchè'' ed il nostro comando precedente non
li sostituirebbe. Se avessimo messo il modificatore \texttt{i} di
case-insensitive il termine ``Perchè'' verrebbe sostituito con ``perché''
perdendo la maiuscola iniziale.
Una prima soluzione è limitarsi alla sostituzione di ``erchè'':
\begin{verbatim}
$ perl -p -i -e 's/erchè/erché/g' chapter/*.tex
\end{verbatim}

Oppure possiamo utilizzare un \emph{gruppo} creato con le parentesi tonde:
\begin{verbatim}
$ perl -p -i -e 's/(P|p)erchè/$1erché/g' chapter/*.tex
\end{verbatim}

In quest'ultimo comando l'espressione regolare \texttt{/(P|p)erchè/} comprende
un carattere iniziale che può essere \texttt{P} oppure \texttt{p} grazie al
carattere \texttt{|}. Poiché la corrispondenza si trova tra parentesi tonde, il
carattere trovato viene restituito nella variabile \texttt{\$1} che andrà a
comporre correttamente la parola da sostituire.

Come avrete notato i comandi sono potenti ma occorre fare molta attenzione
perché potremo non accorgerci di aver sostituito occorrenze che in realtà non
avrebbero dovuto esserlo. Per esempio la parola da sostituire potrebbe essere
parte di altre parole più lunghe.

\subsubsection{I numeri decimali in ambiente matematico}

Nell'ambiente matematico di \TeX{} la virgola produce l'aggiunta di un piccolo
spazio. Questo si può evitare inserendo la virgola in un gruppo per
rappresentare correttamente i numeri decimali: scrivendo \verb!\( v = 12,00 \)!
si ottiene \( v = 12,\, 00 \) mentre con \verb!\( v = 12{,}00 \)! si ottiene \(
v = 12,00 \).

Racchiudere la virgola di separazione decimale in un gruppo, fu la soluzione
adottata in un vecchio progetto. Ora invece vogliamo riutilizzare quei file
sorgenti risolvendo con il pacchetto \packstyle{siunitx} ed inserendo i numeri
decimali in una macro \cs{num}.

Il pattern da ricercare consiste in due sequenze di numeri separate da
\texttt{\{,\}}, ovvero \verb=\d+{,}\d+= dove il simbolo \verb=\d= significa una
singola cifra (e sta per decimal), l'indicatore \texttt{+} indica una o più
ripetizioni. Il comando completo è dunque:
\begin{verbatim}
$ perl -p -i -e 's/(\d+){,}(\d+)/\\num{$1.$2}/g' *.tex
\end{verbatim}



\subsubsection{Sostituire una parola con una macro}

In un progetto suddiviso in molti file sorgenti abbiamo digitato molte volte il
termine Metapost --- il programma di grafica vettoriale --- ma solo adesso ci
siamo accorti che esiste il pacchetto \packstyle{mflogo} che mette a
disposizione la macro \cs{MP}.

Un comando potrebbe essere:
\begin{verbatim}
$ perl -p -i -e 's/metapost/\\MP{}/ig' *.tex
\end{verbatim}
che sostituirebbe nei sorgenti tutte le occorrenze di ``Metapost'' oppure
``METAPOST'' oppure ancora di ``metapost'' con la macro \cs{MP}\texttt{\{\}}.

Qualcosa di più si potrebbe fare inserendo le doppie graffe solo quando
necessario per salvaguardare lo spazio che altrimenti \TeX{} assorbirebbe, con
questi due comandi:
\begin{verbatim}
$ perl -p -i -e 's/metapost(\s+|\w)/\\MP{}$1/ig' *.tex
$ perl -p -i -e 's/metapost(\W)/\\MP$1/ig' *.tex
\end{verbatim}

Il primo pattern individua solamente le occorrenze in cui al termine
``metapost'' segue uno o più caratteri spazio oppure un carattere alfabetico,
il secondo individua solamente le occorrenze in cui al termine ``metapost''
segue un carattere non alfabetico.

Una tabella delle sostituzioni è la seguente:
\begin{Verbatim}[fontsize=\small]
...Metapost è un programma         -> ...\MP{} è un programma...
...MetapostMetapostMetapost<invio> -> ...\MP{}\MP\MP{}<invio>
...lo sviluppo di Metapost.        -> ...lo sviluppo di \MP.
\section{Viva Metapost}            -> \section{Viva \MP}
\end{Verbatim}

I comandi si possono assemblare in un unico comando separando le espressioni di
sostituzione con punti e virgola (il carattere di slash può essere usato per
spezzare su più righe un comando molto lungo per renderlo più leggibile):
\begin{verbatim}
$ perl -pi -e \
 's/metapost(\s+|\w)/\\MP{}$1/ig; s/metapost(\W)/\\MP$1/ig' \
 *.tex
\end{verbatim}

\subsubsection{Righe magiche}

Le righe magiche iniziano con il carattere \texttt{\%} perciò sono ignorate
durante la compilazione, mentre invece sono elaborate da alcuni shell editor
come TeX Works per impostare la codifica del testo, selezionare il programma di
composizione e dichiarare il nome del file sorgente che dovrà essere compilato
effettivamente. Ciò risulta utilissimo per lavorare a progetti complessi in cui
ci sono più file sorgenti separati in diverse cartelle.

Le righe magiche di questo stesso sorgente sono le seguenti:
\begin{verbatim}
% !TEX encoding = UTF-8
% !TEX program = pdflatex
% !TEX root = ../guidaconsole.tex
\end{verbatim}
L'editor lancerà la compilazione con \prog{pdflatex} non del file attualmente
caricato ma del sorgente principale chiamato \texttt{guidaconsole.tex} salvato
nella cartella superiore (sono presenti infatti i due punti \texttt{..}).

Ci proponiamo di modificare il nome del sorgente principale in tutti i file
presenti nella cartella \texttt{chapter} con un comando RE. Se supponiamo come
in questo caso che il nome del file contenga solo caratteri alfanumerici il
pattern che lo rappresenta è \texttt{\textbackslash w+}. Dovremo inoltre
scrivere \texttt{\textbackslash .} al posto del punto semplice \texttt{.} e
\texttt{\textbackslash /} al posto di \texttt{/} perché sono entrambi
metacaratteri:
\begin{Verbatim}[fontsize=\small]
perl -p -i -e 's/(\.\.\|\.)/\w+\.tex/$1\/main.tex/' chapter/*.tex
\end{Verbatim}

Ora, invece di correggere una riga magica già presente nei file, vorremmo che
siano aggiunte ad una collezione di sorgenti \texttt{.tex} posizionati nella
cartella \texttt{cap} della nostra tesi.

Poiché il carattere \verb=^= simboleggia l'inizio della linea potremo
impartire dalla cartella principale i tre comandi seguenti ciascuno col il
compito di inserire una riga magica come prima riga:
\begin{Verbatim}[fontsize=\small]
perl -0777 -pi -e 's/^/% !TEX root = ..\/tesi.tex\n\n/' cap/*.tex
perl -0777 -pi -e 's/^/% !TEX program = pdflatex\n/' cap/*.tex
perl -0777 -pi -e 's/^/% !TEX encoding = UTF-8\n/' cap/*.tex
\end{Verbatim}

L'opzione \texttt{-0777} fa in modo che l'intero file sia trattato come
un'unica linea impostando un carattere di separazione inesistente altrimenti i
caratteri di sostituzione verrebbero aggiunti per ogni inizio di riga. Inoltre
dobbiamo dare le stringhe di sostituzione nell'ordine inverso a quello
desiderato perché le successive si inseriranno in testa --- non che l'ordine
importi nel caso delle righe magiche.

Nel prossimo capitolo vedremo un altro metodo per inserire le righe magiche
all'inizio dei file sorgenti basato sullo scripting di sistema.

