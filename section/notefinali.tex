% !TEX encoding = UTF-8
% !TEX program = pdflatex
% !TEX root = ../guidaConsole.tex

\chapter{Note su questa guida}

\section{Licenza d'uso}

Questo documento è rilasciato con licenza \cc{} \href{http://creativecommons.org/licenses/by-nc-sa/2.5/it/}{Creative Commons Attribuzione - Non commerciale - Condividi allo stesso modo 2.5 Italia License}.

Tu sei libero di riprodurre, distribuire, comunicare al pubblico, esporre in pubblico, rappresentare, eseguire e recitare quest'opera e di modificare quest'opera alle seguenti condizioni:

\begin{description}

\item[\ccby{} Attribuzione:] Devi attribuire la paternità dell'opera nei modi indicati dall'autore o da chi ti ha dato l'opera in licenza e in modo tale da non suggerire che essi avallino te o il modo in cui tu usi l'opera.

\item[\ccnc{} Non commerciale:] Non puoi usare quest'opera per fini commerciali.

\item[\rule{4pt}{0pt}\ccsa{} Condividi allo stesso modo:] Se alteri o trasformi quest'opera, o se la usi per crearne un'altra, puoi distribuire l'opera risultante solo con una licenza identica o equivalente a questa.

\end{description}

\section{Colophon}

Questa guida è stata composta con \LaTeX{}, utilizzando il compositore pdftex con la classe \textsf{guidatematica} appositamente predisposta dal \GuIT* e rilasciata sotto la licenza LaTeX Project Public Licence (LPPL) come enunciata in \url{http://www.latex-project.org/lppl.txt}.

Per presentare \emph{graficamente} procedure ed avvertimenti si è utilizzato il pacchetto \textsf{tcolorbox} e per le combinazioni di tasti e simili il pacchetto \textsf{menukeys}. I loghi per la licenza Creative Commons sono composti con il pacchetto \textsf{cclicenses} di Gianluca Pignalberi.

Dal punto di vista della gestione delle revisioni, i sorgenti della guida sono stati memorizzati con \texttt{git} --- naturalmente con comandi dalla console --- prima verso un repository privato su \url{http://bitbucket.org} e poi nel nuovo repository del \GuIT*{} su \url{https://github.com/GuITeX}, così da risolvere allo stesso tempo le problematiche di sicurezza dei dati e quelle relative all'evoluzione nel tempo dei contenuti.

\section{Collaborazione e ringraziamenti}

Qualsiasi contributo o suggerimento può essere inviato al mio indirizzo di posta elettronica \texttt{giaconet dot mailbox at gmail dot com}.

I sorgenti sono stati resi pubblici per uno sviluppo completamente collaborativo ed aperto nel repository \url{https://github.com/GuITeX/guidalineadicomando}. In particolare è la sezione avanzata della guida che potrebbe ricevere i contributi maggiori e più interessanti.

Ringrazio i lettori che vorranno migliorare questa guida segnalando errori o contribuendo ad estenderne i contenuti. Intanto l'hanno già fatto \textsf{ansys}, \textsf{OldClaudio}, \textsf{Elrond} e Marco Rocco. Grazie mille!

Un ringraziamento particolare va a Claudio Beccari che ha messo a punto la classe per le guide tematiche condividendone lo sviluppo con noi ed ha revisionato frase per frase le varie versioni di questa guida.

Inoltre ringrazio Agostino De Marco per avermi segnalato il sito web \url{http://bitbucket.org} che offre repository \texttt{git} pubblici ma anche privati (spero che un servizio del genere tornerà utile quando scriveremo insieme il prossimo articolo per \Ars, la rivista del \GuIT*), Massimiliano Dominici e tutti i membri del progetto GuITeX  \url{https://github.com/GuITeX?tab=members} per aver creato questo nuovo spazio \emph{virtuale} di collaborazione e diffusione.



% eof
