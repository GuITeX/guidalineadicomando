% !TEX encoding = UTF-8
% !TEX program = pdflatex
% !TEX root = ../guidaconsole.tex

\chapter{Scripting}
\label{chapScripting}

In questa sezione ci addentreremo nella programmazione di shell a cui daremo
la seguente definizione:

\begin{tcolorbox}[title=Definizione di \emph{Scripting}]
Lo Scripting è la scrittura di codice secondo un linguaggio dinamico che verrà
eseguito direttamente da un interprete
\end{tcolorbox}

Quello che caratterizza lo scripting rispetto all'attività alla riga di
comando descritta fino ad ora è dunque la scrittura di un vero e proprio
\emph{programma} il cui codice dovrà essere conforme ad uno specifico
linguaggio che normalmente non prevede la dichiarazione preventiva del tipo di
dato che viene invece stabilito \emph{dinamicamente} durante l'esecuzione.

Per esempio, l'assegnazione di un intero e di una stringa è qualcosa di simile
al seguente frammento di codice:
\begin{Verbatim}
n = 10         % invece di:  int n = 10
s = "dieci"    % invece di:  string s = "dieci"
\end{Verbatim}

L'esecuzione del codice viene affidata ad un \emph{interprete} che lo
elabora direttamente secondo una procedura anche molto sofisticata e che rende
davvero minima la differenza tra i moderni linguaggi di scripting e quelli
compilati il cui codice viene invece tradotto in una forma binaria
memorizzata in un file oggetto che sarà poi l'eseguibile effettivo.
Questa sempre più sfumata distinzione tra linguaggi interpretati e compilati è
dovuta al fatto che molto spesso anche i primi traducono il sorgente in
una forma binaria intermedia per ambienti di esecuzione detti \emph{macchine
virtuali}. Inoltre la tipizzazione dinamica dei dati non è un'esclusiva dei
linguaggi di scripting. Pensato per la programmazione di sistema il
\href{http://golang.org/}{Go}, l'ambiente open~source di sviluppo di Google,
prevede l'operatore \texttt{:=} per l'assegnamento dinamico.

La shell del sistema operativo in uso offre generalmente un linguaggio di
scripting predefinito. Non vi è alcuna differenza concettuale se il codice
viene digitato direttamente al prompt dei comandi oppure letto da un file di
testo --- che assume il nome di \emph{script}.

I linguaggi di scripting più conosciuti sono
\href{http://www.python.org/}{Python}, \href{http://www.lua.org/}{Lua},
\href{http://www.perl.org/}{Perl}, \href{http://www.ruby-lang.org/}{Ruby},
\href{http://www.falconpl.org/}{Falcon},
\href{http://it.wikipedia.org/wiki/JavaScript}{Javascript}, eccetera. Quasi
sempre si tratta di software libero disponibile per le piattaforme desktop più
diffuse tra cui ovviamente i sistemi Windows, Linux ed OS~X.

\section{Sha bang\#!}

Abbiamo lasciato in sospeso la risposta ad una domanda che forse è balenata al
lettore: visto che esistono molti linguaggi di scripting, come fa la shell al
momento dell'esecuzione di uno script a chiamare l'interprete giusto?

Per tutti i sistemi ed in particolare per quelli Windows, l'informazione che
lega lo script all'interprete per cui è stato scritto è fornita dall'utente in
modo esplicito, ovvero lo script deve essere lanciato da riga di comando dando
il nome del file che lo contiene come argomento dell'interprete.

Oppure, può essere sufficiente un doppio click sull'icona del file di script
se l'estensione è stata associata al corrispondente eseguibile dell'interprete
tramite il riconoscimento automatico degli ambienti ad interfaccia grafica di
Linux e Mac od il registro di sistema in Windows.

Sui sistemi di tipo Unix --- come Linux ed OS~X --- la risposta alla domanda
iniziale può tuttavia essere molto semplice: si inserisce il riferimento
all'interprete corretto all'interno dello script stesso --- generalmente nella
prima riga --- ed è proprio quello che rappresenta la così detta
\emph{sha~bang}, formata dalla coppia di caratteri \texttt{\#!} seguiti dal
percorso dell'eseguibile\footnote{Se non fosse standard la struttura delle
  directory principali di un sistema di tipo Unix la sha~bang di uno script
  dovrebbe adattarsi al sistema operativo.}. Quasi tutti i linguaggi di
scripting supportano la sha~bang nel senso che si tratta di una riga contenente
informazioni per la shell che non deve essere interpretata ma perfettamente
ignorata come se fosse un commento.

Con questo trucco il nome del file dello script diventa un comando,
esattamente come quelli di sistema compilati. Per di più nella maggior parte
dei linguaggi è prevista la possibilità di elaborare gli argomenti digitati
dall'utente al terminale.

Immaginiamo uno script in linguaggio Lua contenuto nel file \texttt{compila}
che accetti un argomento. Per eseguirlo possiamo dare il comando:
\begin{Verbatim}
$ lua compila dir  % se il sistema non supporta la sha bang
$ compila dir      % se invece si
\end{Verbatim}

\section{Esempi}

Terminata la presentazione dei concetti basilari dello scripting, possiamo
scrivere i nostri script personali utili ad automatizzare processi o a produrre
vere e proprie applicazioni. L'argomento è vastissimo: possiamo pensare di
scrivere semplici sequenze di comandi oppure creare complessi programmi ad
interfaccia grafica, scegliendo tra molti ed efficienti linguaggi e librerie.

% oi_begin_comment
% Credo anche che sia necessario un approfondimento su CygWin
% oi_end_comment
\subsection{Ancora sulle righe magiche}
\label{sssec:addheader}

Supponiamo di aver scritto diverso tempo fa dei sorgenti \LaTeX, quando ancora
non conoscevamo l'editor TeX~Works. In quei file ovviamente non sono presenti le
righe magiche che TeX~Works riesce ad interpretare impostando codifica e
programma di composizione, ed oggi, a distanza di tempo, vorremmo aggiungerle a
ciascuno di quei file \texttt{.tex}.

Per questo inseriamo in una cartella i sorgenti ed un file che contenga le
righe magiche di cui abbiamo bisogno, per esempio queste:
\begin{Verbatim}
% !TEX encoding = UTF-8
% !TEX program = pdflatex
\end{Verbatim}
e salviamo questo file con il nome \filestyle{header} (possiamo anche non
assegnare nessuna estensione). A questo punto apriamo la shell e scriviamo la
riga\footnote{Per una questione di praticità in questo documento il codice è
stato spezzato su più righe.}
\begin{Verbatim}
for i in *.tex; do \
 cp header header_temp; \
 cat $i >> header_temp; \
 mv header_temp $i; \
done
\end{Verbatim}
e vedremo come, premendo il tasto \keys{\return}, a tutti i file sarà stato
aggiunto quanto scritto nel file \filestyle{header}.

Si tratta di codice per Bash che risolve in altro modo un problema già
incontrato tra i comandi avanzati del capitolo precedente. Il codice è molto
semplice: con \verb!cp header header_temp;! si crea una copia del file che
contiene le righe magiche; con \verb!cat $i >> header_temp;! uniamo il
contenuto del file \texttt{.tex} che la variabile \verb!$i! sta
scorrendo, con il file \filestyle{header\_temp}; infine con
\verb!mv header_temp $i;! rinominiamo \filestyle{header\_temp} con il nome del
file \texttt{.tex} a cui abbiamo aggiunto le righe sovrascrivendolo.

Per affinamenti successivi si può sicuramente ottenere un codice più efficiente
(e questo forse non lo è, ma sicuramente raggiunge il suo scopo), da salvare in
un file di estensione \texttt{.sh} che possa, per esempio, lavorare su file
presenti in sotto-cartelle, oppure verificare che tali righe siano già presenti
o meno.

Quanto scritto può essere fatto ovviamente su sistemi operativi Linux e Mac OS
X con la shell di sistema; per i sistemi operativi Windows tale procedura è
possibile a patto di installare \href{http://www.cygwin.com/}{Cygwin} o un
altro emulatore Unix.

\subsection{Upload via ftp}

Una caso che riguarda proprio le guide tematiche: vorremmo mettere a punto uno
script che esegua la connessione sicura al sito del \GuIT*{} e provveda a
caricarvi uno o più documenti \texttt{pdf} in modo da renderli disponibili per
il download.

Operativamente, l'esecuzione di uno script siffatto è molto più immediata che
non utilizzare un programma ad interfaccia grafica che supporti il protocollo
\texttt{ftp} (\emph{file transfert protocol}). Definiamo dapprima la sintassi
del comando \texttt{curl} a cui ci affidiamo per le transazioni \texttt{ftp} e
poi creiamo lo script con il linguaggio di shell.
\begin{Verbatim}
$ curl -u <user>:<password> -T <file> <url destination path>
\end{Verbatim}

Il programma \href{http://curl.haxx.se/}{curl} supporta un notevole numero di
protocolli di rete. L'opzione \texttt{-u} inserisce le credenziali dell'utente
presso il server, l'opzione \texttt{-T} indica che desideriamo procedere al
caricamento del file verso la destinazione remota. Il formato della destinazione
definisce anche il tipo di connessione che desideriamo stabilire che nel caso in
esame sarà una connessione \texttt{ftp}.

Dopo aver indicato che si tratta di uno script per la shell di
sistema\footnote{Nei moderni sistemi Unix \texttt{sh} è un collegamento ad una
  shell standard \href{http://it.wikipedia.org/wiki/POSIX}{POSIX} con cui
  possono essere prodotti script in grado di essere eseguiti correttamente su
  diversi sistemi operativi della famiglia. Se si desidera usufruire delle
  funzionalità aggiuntive di alcune shell come \texttt{bash}, la Bourne Again
  Shell, al prezzo di perdere la portabilità, è sufficiente utilizzare la
  sha~bang \texttt{\#!/bin/bash}.  Le distribuzioni Debian e derivate come
  Ubuntu, per motivi di efficienza preferiscono assegnare \texttt{sh} alla
  shell \texttt{Dash}, la Debian Almquist shell, più snella e performante
  specie nell'accorciare i tempi di avvio.} per mezzo della Sha~Bang, definiamo
le variabili per il percorso di destinazione sul server remoto così da
facilitarne la modifica e, per mezzo di un ciclo esteso a tutti i file
\texttt{pdf} contenuti nella cartella di lavoro, eseguiamo uno alla volta
l'upload dei documenti delle guide.

Di seguito è riportato il codice dello script. Si noti come l'uso delle
variabili avviene con una modalità di espansione premettendo il carattere
dollaro al nome.
\begin{Verbatim}
#!/bin/sh
guit_ftp='ftp://ftp.guitex.org/'
guit_dir='httpdocs/home/images/doc/GuideGuIT/'
for f in *.pdf
do
   echo "Upload del file :$f"
   curl -T $f $guit_ftp$guit_path$f
   echo "Fatto!"
done
\end{Verbatim}

Abbiamo trascurato di inserire le informazioni delle credenziali perché
immaginiamo che esse siano reperibili automaticamente nell'opportuno formato
nel file di configurazione di \texttt{curl} di nome \texttt{.curlrc} che si
trova nella directory \texttt{home} dell'utente. Si sono tralasciati i
problemi di sicurezza relativi alla conservazione e alla trasmissione sicura
delle credenziali a favore della semplicità.

Salvato il codice in un file chiamato \texttt{upguide} assegnamogli i
diritti di esecuzione con il comando:
\begin{Verbatim}
$ chmod +x upguide
\end{Verbatim}
A questo punto il suo utilizzo è semplice: basterà avviarlo una volta impostata
la cartella di lavoro a quella contenente i documenti da caricare sul sito.

Potremo voler aggiungere allo script l'invio di una e-mail ad un elenco di
interessati per comunicare la disponibilità di aggiornamenti\dots

\subsection{Uno script con Lua}

Come esempio di semplice script in un linguaggio diverso da Bash, consideriamo
il codice Lua seguente. Lo scopo è stampare a video alcune informazioni centrate
sulla riga di 80 caratteri, considerando la possibilità che l'utente specifichi
quale argomento opzionale una lunghezza di riga diversa. I valori degli
argomenti sono disponibili all'interno dello script nella tabella predefinita di
nome \texttt{arg}:
\begin{Verbatim}
#!/usr/bin/lua
ll = arg[1] or 80

local function prt( s )
   local n = (ll - #s)/2
   local sp = string.rep(' ', n)
   print(sp..s)
end

prt "Titolo:  Guida tematica alla riga di comando"
prt "Collana: Le guide tematiche del GuIT"
prt "..."
\end{Verbatim}
Al solito, per costruire lo script, salviamo il codice nel file
\texttt{centermsg} rendendolo eseguibile con \texttt{chmode} se siamo in
ambiente Unix, e digitiamo alla riga di comando:
\begin{Verbatim}
$ ./centermsg    # in Linux o Mac
> lua centermsg  # in Windows
\end{Verbatim}
Ricordo che in Lua se una funzione, come in questo caso \texttt{prt}, prevede
un unico argomento di tipo stringa o tabella, possiamo omettere le parentesi
tonde.

% contributo dal forum del GuIT di Marco87 e Francesco Biccari
% http://www.guitex.org/home/it/forum/6-altri-programmi/81066-risolto-crop-e-scontorno
% per il file batch per Windows con pdfcrop
\subsection{Ritagliare le immagini}
\label{subsec}

Abbiamo incontrato il programma \texttt{pdfcrop} nella
sezione~\ref{secInstVer} della guida, per ritagliare in modo automatico i bordi
bianchi --- privi di contenuto --- da un documento \texttt{pdf}. Il codice
seguente, se inserito in un file di estensione \texttt{.bat}, diventa uno
script per Windows che scontorna tutti i file \texttt{pdf} presenti nella
cartella di lavoro conservando una copia dell'originale con il suffisso
\texttt{\_old}:
\begin{Verbatim}
@echo off
echo Cropping...
FOR %%A IN (%*) DO (
   echo %%A
   IF /I %%~xA==.pdf (
      copy /Y %%A "%%~nA_old.pdf"
      cd "%%~dA%%~pA"
      pdfcrop.exe --margins 1 "%%~nA_old.pdf" "%%~nA.pdf"
   )
)
\end{Verbatim}

Notare come mentre nella shell di Unix i blocchi di comandi sono individuati
da una chiave finale come \texttt{end} o \texttt{done}, nel linguaggio di
scripting di Windows sono invece racchiusi da parentesi tonde, ma la sostanza
non cambia di molto.

Una scorciatoia per l'avvio dello script su una selezione di file eseguita con
il mouse, è quella di salvare il file corrispondente nella cartella che si
apre premendo il tasto \keys{Windows} e digitando nella casella di testo
\texttt{shell:sendto} \keys{\return}. Il nome dello script comparirà nel menù
contestuale pronto a lavorare sui file selezionati.

Anche su Linux è possibile fare una cosa simile, inserendo i file degli script
nella cartella \texttt{home/.gnome2/nautilus-script}, così che appariranno
altrettante voci nel menù ``Script'' --- almeno per Gnome con il suo gestore
di finestre Nautilus. Agli script verranno passati i nomi dei file selezionati
via mouse attraverso normali variabili di shell dai nomi prefissati.

Per esempio, uno script Lua che elaborasse questa lista di file potrebbe
contenere il seguente codice:
\begin{Verbatim}
-- ottengo la lista dei file
local s = os.getenv("NAUTILUS_SCRIPT_SELECTED_FILE_PATHS")

-- elaboro i singoli file
for path in string.gmatch(s,"(.-)\n") do
-- qualcosa con il file di percorso <path>
-- ...
end
\end{Verbatim}

Con la shell l'elenco dei file è disponibile anche nella variabile
\texttt{arg}. Uno script che volesse elencare i file della selezione in un
file di testo e comunicare all'utente in una finestra grafica sfruttando
\href{http://live.gnome.org/Zenity}{Zenity} ---  una libreria multipiattaforma
che permette l'esecuzione di finestre di dialogo GTK+ --- il messaggio con il
loro numero, potrebbe essere il seguente:
\begin{Verbatim}
#!/bin/sh

for arg do
    echo "$arg"
done >> listoffiles.txt

zenity --info --width=200 --height=100 \
       --title="Conteggio files elencati" \
       --text="Numero di file elencati: $# "
\end{Verbatim}

Ricordatevi di dare i permessi di esecuzione ai file di script altrimenti non
risulteranno nemmeno tra le voci del menù contestuale dell'ambiente grafico.



% end of file :-)
