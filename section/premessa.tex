% !TEX encoding = UTF-8
% !TEX program = pdflatex
% !TEX root = ../guidaconsole.tex

%
%
%
\chapter{Piano della guida}

Questa guida tematica è suddivisa in quattro sezioni principali:
\begin{enumerate}
\item è una presentazione concettuale della riga di comando sostenuta da accenni
alla sua lunga storia (pagina~\pageref{chapConsole});

\item è un sunto delle particolarità operative di base dell'ambiente di shell
(pagina~\pageref{chapShell});

\item è lo svolgimento passo passo delle procedure per raggiungere le capacità e
le conoscenze necessarie per lavorare autonomamente con la shell di sistema, da
leggere anche indipendentemente dalle altre sezioni (pagina~\pageref{chapEser});

\item è un compendio di argomenti avanzati (pagina~\pageref{chapAvanz})
assieme alla presentazione delle tecniche di \emph{scripting}
(pagina~\pageref{chapScripting}).
\end{enumerate}

La modalità di consultazione dipende dalle conoscenze e dagli obiettivi del
lettore. Probabilmente chi già utilizza la shell sarà interessato ad uno sguardo
sugli aspetti generali ed ad un approfondimento su quelli specifici per
l'utilizzatore \TeX, mentre il neofita potrebbe volersi cimentare subito con
l'indirizzo pratico della terza sezione e sorvolare sui concetti che la
sovrintendono.

Comunque sia, auguro a tutti una buona e spero proficua lettura. Scrivetemi
senza indugio un messaggio di posta elettronica per segnalarmi errori o
miglioramenti o richieste di chiarimenti. Spero mi mandiate le vostre
impressioni, soprattutto se non avevate mai lavorato con la riga di comando
prima d'ora \texttt{:-)} .

\medskip
\hfill\begin{tabular}{c}
Roberto Giacomelli\\
\texttt{giaconet dot mailbox at gmail dot com}\\
\end{tabular}
% eof
