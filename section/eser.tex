% !TEX encoding = UTF-8
% !TEX program = pdflatex
% !TEX root = ../guidaconsole.tex

\chapter{Esercitazioni}
\label{chapEser}

Le esercitazioni sono applicazioni pratiche orientate all'utilizzo del sistema
\TeX{} dalla riga di comando, e basate sui concetti di funzionamento della
shell. Mettere in pratica i comandi illustrati richiede di conoscere il modo in
cui si avvia la shell sul proprio computer (in caso consulta la
sezione~\ref{secAvvio} di pagina~\pageref{secAvvio}).

\section{Cambiare directory di lavoro}

Su tutti i sistemi il comando per cambiare la directory di lavoro della shell è
\texttt{cd} che sta per \textbf{c}hange \textbf{d}irectory:

\begin{tcolorbox}
\begin{verbatim}
$ cd <path>
\end{verbatim}
\end{tcolorbox}

La sintassi prevede di specificare come primo argomento il percorso della
directory. Il comando dato senza argomenti, sui sistemi Unix riporta la
directory di lavoro alla \texttt{home} dell'utente, su quelli Windows invece,
stampa semplicemente la directory di lavoro attuale, per altro informazione già
riportata nel prompt (in Unix la stampa della directory corrente si ottiene con
il comando \texttt{pwd}, \textbf{p}rint \textbf{w}orking \textbf{d}irectory).

Nella digitazione dei path è molto comodo avvalersi del \emph{completamento
automatico}, che funziona premendo il tasto \keys{Tab} dopo aver digitato un
numero sufficiente di caratteri da individuare univocamente una directory, così
da non dover più digitarne il nome per intero.

Provare ad impostare la directory di lavoro al valore di \verb=/usr/bin= in
Linux e \verb=C:\Document and Settings= in Windows XP o \verb=C:\Users= in
Windows 7.

Per risalire al nodo superiore del file system digitare poi il comando
\verb=$ cd ..=

\section{Elencare i file}

Una delle necessità più frequenti dell'utente è quella di stampare a video
l'elenco delle cartelle e dei file presenti in un nodo del file system. In
Windows il comando per farlo è \texttt{dir}, mentre negli altri sistemi è
\texttt{ls} (\textbf{l}i\textbf{s}t file). Apriamo una finestra di console e
digitiamo il comando senza argomenti per mostrare il contenuto della directory
di lavoro.

\noindent\begin{tcolorbox}[width=(\linewidth-6pt)/2,before=,after=\hfill]
Windows\tcblower
\begin{verbatim}
> dir
\end{verbatim}
\end{tcolorbox}
\begin{tcolorbox}[width=(\linewidth-6pt)/2,before=,after=\hfill]
Linux e Mac OS X\tcblower
\begin{verbatim}
$ ls
\end{verbatim}
\end{tcolorbox}

Possiamo utilizzare filtri per elencare file corrispondenti a particolari
criteri. Il carattere asterisco \texttt{*} sta per qualsiasi sequenza di
caratteri. Il carattere \texttt{?} sta per un qualsiasi singolo carattere. Alla
riga di comando proviamo quindi ad elencare i sorgenti \texttt{.tex} (questa è
l'estensione standard per i nomi dei file sorgenti del sistema \TeX):

\noindent\begin{tcolorbox}[width=(\linewidth-6pt)/2,before=,after=\hfill]
Windows\tcblower
\begin{verbatim}
> dir *.tex
\end{verbatim}
\end{tcolorbox}
\begin{tcolorbox}[width=(\linewidth-6pt)/2,before=,after=\hfill]
Linux e Mac OS X\tcblower
\begin{verbatim}
$ ls *.tex
\end{verbatim}
\end{tcolorbox}

Naturalmente è possibile elencare i file contenuti in qualsiasi directory
specificandone il percorso completo indipendentemente dalla directory di lavoro.

Ecco una sessione di lavoro in Linux per elencare i file nella directory
principale dei contenuti di questa guida. Per prima cosa si imposta la directory
di lavoro e poi si chiede la stampa della lista dei file contenuti poi, rilevata
la presenza di una sotto cartella \texttt{section}, si chiede l'elenco dei
sorgenti \TeX{} a sua volta in essa contenuti:

\begin{Verbatim}[fontsize=\small]
$ cd Scrivania/guidaConsole/
$ ls
guidaConsole.aux  guidaConsole.pdf         guidaConsole.toc
guidaConsole.log  guidaConsole.synctex.gz  image
guidaConsole.out  guidaconsole.tex         section

$ ls section/*.tex
section/argavanz.tex  section/intro.tex       section/premessa.tex
section/eser.tex      section/notefinali.tex  section/shell.tex
\end{Verbatim}

\section{Spostare o copiare file}

Per spostare i file tra una directory e l'altra, si utilizza il comando
\texttt{mv} (\textbf{m}o\textbf{v}e), specificando come argomenti prima il
percorso del file attuale poi quello di destinazione. In Windows il comando si
chiama \texttt{move}.
\smallskip
\begin{tcolorbox}
Windows
\tcblower
\begin{verbatim}
> move <source path> <destination path>
\end{verbatim}
\end{tcolorbox}

\begin{tcolorbox}
Linux e Mac OS X
\tcblower
\begin{verbatim}
$ mv <source path> <destination path>
\end{verbatim}
\end{tcolorbox}

Se il percorso di destinazione punta alla stessa directory di partenza, il
comando \texttt{mv} rinomina il file.

Per copiare anziché spostare i file si usa il comando \texttt{cp} in Linux e
Mac e il comando \texttt{copy} in Windows, specificando al solito prima i file
da copiare e poi la loro nuova destinazione.

\section{Rinominare file}

Cambiare nome ad un file è un operazione concettualmente equivalente allo
spostamento perché in entrambe i casi si tratti di modificarne il percorso.
Torna utile specie se vi trovate ad utilizzare Windows che solitamente è
impostato\footnote{\`E preferibile non farlo ma per cambiare questa opzione
cliccare col destro sulla finestra grafica dell'Explorer e dal dialogo delle
proprietà togliere la spunta su ``Nascondi l'estensione dei file''.} per
nascondere la estensioni dei file, rendendo impossibile cambiarla in modalità
grafica.

Il comando da usare è \texttt{rename} con la stessa sintassi per tutti i
sistemi:

\begin{tcolorbox}
\begin{verbatim}
$ rename <old_name> <new_name>
\end{verbatim}
\end{tcolorbox}

Per esempio, rinominare il file \texttt{uno.tex} --- presente nella directory di
lavoro ---, in \texttt{due.tex}, l'istruzione da riga di comando è:
\begin{verbatim}
$ rename uno.tex due.tex
\end{verbatim}

\section{Verificare l'installazione \TeX}
\label{secInstVer}

Dopo l'installazione di una distribuzione \TeX{} è possibile verificare il
funzionamento dei programmi con semplici comandi da console. Per prima cosa ci
si chiede se il sistema riconosce gli eseguibili dei motori di composizione. In
altre parole, il \texttt{PATH} di sistema dovrà contenere il percorso degli
eseguibili, che appositamente per questo sono contenuti tutti in un'unica
directory.

Per esempio, il seguente comando avrà successo se gli eseguibili sono
correttamente installati e configurati:
\begin{verbatim}
$ pdftex -version
\end{verbatim}

Compiliamo adesso un sorgente \TeX{} veramente minimo che contiene poche parole,
inserendo con un editor di testi questo codice in un file, naturalmente di
estensione \texttt{.tex}, che chiameremo \texttt{testtex.tex}:
\begin{verbatim}
Ciao Mondo da \TeX!
\bye
\end{verbatim}

Una volta impostata la directory di lavoro in modo che coincida con quella
contenente il sorgente di prova, il comando di compilazione per la verifica di
funzionamento è:
\begin{verbatim}
$ tex testtex
\end{verbatim}

Otteniamo un file di uscita con estensione \texttt{dvi} \emph{Device
Indipendent}, il formato classico di \TeX{} che --- l'occasione è buona per
verificare il funzionamento di alcuni utili programmi compresi nelle
distribuzioni e quindi già in dotazione --- convertiamo in \texttt{pdf}
\emph{Portable Document Format} con il tradizionale doppio passaggio, il primo
da \texttt{dvi} al formato \texttt{ps} \emph{Postscript} con il driver
\texttt{dvips} ed il secondo da Postscript a \texttt{pdf}:
\begin{verbatim}
$ dvips testtex     -- conversione dvi --> ps
$ ps2pdf testtex.ps -- conversione  ps --> pdf
\end{verbatim}

Oggi questi passaggi di formato si utilizzano raramente perché si lavora
direttamente e convenientemente con il formato di uscita \texttt{pdf}, come
faremo tra poco con \texttt{pdflatex}.

A questo punto dovreste aver correttamente ottenuto i messaggi di compilazione
in console e tutti i file di uscita previsti. Non rimane che verificare il
funzionamento di \LaTeX{} con questo piccolo sorgente da inserire in un file di
testo che chiameremo \texttt{testlatex.tex}:
\begin{Verbatim}[fontsize=\small]
\documentclass{article}
\usepackage[T1]{fontenc}
\usepackage[utf8]{inputenc}
\usepackage[italian]{babel}

\begin{document}
   Ciao Mondo da \LaTeX!
\end{document}
\end{Verbatim}

Il comando di compilazione sarà quindi il seguente che produrrà in uscita il
file nel formato \texttt{pdf}:
\begin{verbatim}
$ pdflatex testlatex
\end{verbatim}

Possiamo completare le verifiche all'installazione provando il corretto
funzionamento dello strumento \texttt{pdfcrop}. Si tratta di un programma a riga
di comando scritto in \texttt{perl} che, appoggiandosi a \textsf{Ghostscript},
ritaglia un documento \texttt{pdf} eliminando i contorni fino a raggiungere le
dimensioni minime del contenuto, operazione che facilita l'inclusione nei
documenti \LaTeX{} dei contenuti \texttt{pdf}.

Per eseguire una prova sul file \texttt{testlatex.pdf} appena creato, impostate
in una sessione di terminale la directory corrente a quella contenente il
documento da scontornare e date il seguente comando al termine del quale
dovreste trovare nella stessa directory del file originale un nuovo file dal
nome \texttt{testlatex-crop.pdf} correttamente scontornato:
\begin{verbatim}
$ pdfcrop testlatex.pdf
\end{verbatim}

Lo strumento utilissimo \texttt{pdfcrop} accetta molte opzioni. Le più
importanti sono \texttt{--margin <valore>}, con cui si imposta un margine
attorno al rettangolo ritagliato esprimendo misure in punti, \texttt{--hires}
per il calcolo in alta risoluzione del \emph{bounding box} che è appunto il
rettangolo minimo che contiene gli oggetti nella pagina, e la possibilità di
impostare il nome del file ritagliato semplicemente scrivendolo di seguito al
nome del file originale.

Per esempio, per scontornare il file di prova precedente lasciando comunque un
margine di 5 punti e sostituendo il file su disco con quello ritagliato, il
comando è il seguente (per comodità utilizzate il completamento automatico
premendo il tasto \keys{Tab} per produrre facilmente i nomi dei file):
\begin{verbatim}
$ pdfcrop --margin 5 testlatex.pdf testlatex.pdf
\end{verbatim}


La riga di comando ci ha permesso di verificare l'installazione del sistema
\TeX. Per completezza occorre aggiungere che alcuni shell editor come Kile per
Linux od il multi-piattaforma TeXWorks, dispongono di un apposita voce di menù
che si occupa di lanciare le compilazioni di verifica della corretta
installazione dei programmi della distribuzione \TeX{} informando l'utente sul
funzionamento del sistema.

\section{Compilare un documento sorgente}

La sintassi generale per compilare un file sorgente con uno dei programmi del
sistema \TeX{} è la seguente, dove il nome del file compare senza estensione:

\medskip
\texttt{\$ \prog{programma-di-composizione} \meta{--opzioni} \meta{nome del
file}}
\medskip

Un sorgente \LaTeX{} può essere compilato sia con il programma \texttt{latex}
sia con \texttt{pdflatex}. La differenza è che con il primo si ottiene il
documento nel formato \texttt{dvi} e con il secondo nel formato \texttt{pdf}.
Per molte buone ragioni è preferibile utilizzare \texttt{pdflatex}:
\begin{tcolorbox}
Windows, Linux e Mac OS X
\tcblower
\ttfamily
\$ \prog{pdflatex} \meta{nomefile senza estensione}
\end{tcolorbox}

\subsection{L'opzione \texttt{-shell-escape}}

Per ragioni di sicurezza del sistema, alcuni pacchetti \LaTeX{} come
\textsf{pgfplots} o \textsf{gmp} richiedono compilazioni con l'opzione
\texttt{-shell-escape} per abilitare il compositore all'esecuzione di programmi
esterni. Coerentemente alla sintassi generale illustrata alla
sezione~\ref{chapConsole}, il comando di compilazione diventa:

\medskip
\texttt{\$ \prog{pdflatex} -shell-escape \meta{nomefile senza estensione}}
\medskip

Questo comando è utile perché a volte si preferisce lanciarlo da riga di comando
piuttosto che configurare l'editor per aggiungere nell'elenco dei comandi di
compilazione questa opzione, anche se solitamente è molto più comodo lanciare il
comando all'interno dell'editor grafico piuttosto che scomodare la console.

\section{Documentarsi con \textsf{texdoc}}

Uno dei programmi più utili di una distribuzione \TeX{} è senza dubbio l'utility
\texttt{texdoc} che permette di accedere rapidamente alla documentazione di un
pacchetto o di un programma di composizione, generalmente in formato
\texttt{pdf}.

Per visualizzare la guida di un pacchetto particolare il comando è:
\begin{tcolorbox}
\ttfamily
\$ \prog{texdoc} \meta{nome pacchetto}
\end{tcolorbox}

Invece, se non si è sicuri del nome del pacchetto o se si cerca documentazione
per un dato argomento, si utilizza l'opzione \texttt{-l} che sta per
\emph{list}, per ottenere l'elenco dei documenti riguardanti una particolare
chiave di ricerca:
\begin{verbatim}
$ texdoc -l chiave
\end{verbatim}

Per esempio provate a cercare con \texttt{texdoc} la documentazione riguardante
Metapost, un linguaggio per il disegno programmato. Per la distribuzione
\TeX{}~Live~2012 l'utility troverà i seguenti documenti visualizzabili dando il
numero corrispondente --- per questo \texttt{texdoc} è un programma con un
minimo di funzionalità interattive:
\begin{Verbatim}[fontsize=\small]
$ texdoc -l metapost
1 /usr/local/texlive/2012/texmf-dist/doc/metapost/base/mpman.pdf
2 /usr/local/texlive/2012/texmf-dist/doc/latex/pdfslide/mpgraph.pdf
3 /usr/local/texlive/2012/texmf-dist/doc/metapost/base/mpgraph.pdf
4 /usr/local/texlive/2012/texmf-dist/doc/metapost/base/mpintro.pdf
Please enter the number of the file to view, anything else to skip:
\end{Verbatim}

Oppure, per consultare il manuale del programma \texttt{pdflatex} con
l'illustrazione della sintassi del comando e di tutte le opzioni possibili, come
l'impostazione del nome del file di uscita (\texttt{-jobname <name>}),
l'impostazione di un formato precompilato per velocizzare la compilazione
(\texttt{-fmt <format>}), l'impostazione del modo di interazione durante la
compilazione (\texttt{-interaction <mode>}) e molto altro ancora, digitate:
\begin{verbatim}
$ texdoc pdflatex
\end{verbatim}

\section{Gestione di \TeX\ Live}

La distribuzione \TeX\ Live scaricabile da \textsc{ctan} è dotata dell'utility
\texttt{tlmgr} (\TeX\ Live manager)\footnote{Alcune distribuzioni Linux come
Debian e quindi Ubuntu, a causa delle politiche di sicurezza e di stabilità del
sistema, non distribuiscono nei propri repository \TeX\ Live con \texttt{tlmgr}.
Per questa ed altre ragioni conviene installare \TeX\ Live direttamente dai
mirror di \textsc{ctan}. Leggendo una buona guida come quella di Enrico Gregorio
scaricabile all'indirizzo \url{profs.sci.univr.it/~gregorio/texlive-ubuntu.pdf}
si possono evitare anche i problemi dovute alle dipendenze di alcuni programmi
come Kile verso \TeX~Live dei repository della distribuzione Linux. Alcune
spiegazioni aggiuntive per gli utenti del pinguino si trovano nei post
\url{http://robitex.wordpress.com/2010/09/15/installare-tex-live-2010-in-ubuntu-lucid-lynx/}
e
\url{http://robitex.wordpress.com/2010/10/12/kile-e-tex-live-2010-su-sistemi-ubuntu/}.}.
Si tratta di uno script in Perl solitamente utilizzato da riga di comando ma di
cui è disponibile anche una versione grafica.

Nei sistemi Mac~OS~X \texttt{tlmgr} è sostituito da un programma ad interfaccia
grafica distribuito con Mac\TeX, la versione di \TeX\ Live per i sistemi Apple,
chiamata \TeX{}~Live Utility.

Ecco una selezione dei comandi più utili per la gestione della distribuzione con
\TeX{}~Live manager (lanciate il comando \verb=$ texdoc tlmgr= per consultare
tutte le opzioni disponibili e le corrispondenti sintassi). Alcuni dei comandi
potrebbero richiedere la modifica di directory di sistema e pertanto è
necessario acquisire i diritti di amministratore:

Aggiornamento di tutti i pacchetti installati:
\begin{verbatim}
$ tlmgr update --all
\end{verbatim}

Elencare i pacchetti non aggiornati:
\begin{verbatim}
$ tlmgr update --list
\end{verbatim}

Aggiornare l'utility \texttt{tlmgr} stessa:
\begin{verbatim}
$ tlmgr update --self
\end{verbatim}

Lanciare \texttt{tlmgr} in modalità grafica:
\begin{verbatim}
$ tlmgr gui
\end{verbatim}

Consultare lo stato di un pacchetto con alcune informazioni di base:
\begin{verbatim}
$ tlmgr show nome_pacchetto
\end{verbatim}

Impostare un repository \textsc{ctan} particolare (digitare il comando su un
unica riga sostituendo a \texttt{<mirror>} il percorso di un server
\textsc{ctan}):
\begin{verbatim}
$ tlmgr option repository <mirror>/tex/systems/texlive/tlnet/
\end{verbatim}

Per esempio un \emph{mirror} piuttosto veloce almeno per la mia connessione è
quello di un'università olandese, ed allora il comando diventa (da digitare su
una stessa riga):
\begin{Verbatim}[fontsize=\small]
$ tlmgr option repository
  ftp://ftp.snt.utwente.nl/pub/software/tex/systems/texlive/tlnet/
\end{Verbatim}



