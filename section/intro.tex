% !TEX encoding = UTF-8
% !TEX program = pdflatex
% !TEX root = ../guidaconsole.tex

\chapter{La riga di comando}
\label{chapConsole}

Prima della comparsa delle interfacce grafiche --- ideate al Palo Alto Research
Center in California ed implementate da Apple prima in Lisa e poi in Macintosh
nel 1983 --- l'interazione con i programmi avveniva con una modalità testuale
chiamata \emph{riga di comando}, ancora oggi disponibile sui moderni elaboratori
come componente essenziale e metodo efficiente di elaborare dati.

I componenti del sistema \TeX{} possono essere utilizzati per mezzo della sola
interfaccia a riga di comando --- ed è per questo che gli utilizzatori
potrebbero avvantaggiarsi se la conoscessero. Ciò non toglie che componenti
indipendenti come gli \emph{shell editor}, possano costituire un'interfaccia
grafica verso i programmi di composizione della famiglia \TeX{}.

\section{Il concetto di base}

\begin{tcolorbox}[title=Definizione di \emph{Riga di comando}]
La \emph{riga di comando} è un ambiente testuale in cui si impartiscono
istruzioni digitando nomi di programmi con eventuali argomenti
\end{tcolorbox}

L'ambiente testuale che consente all'utente di interagire con il sistema è
chiamato \emph{shell}. La shell accetta i dati di ingresso sotto forma di
\emph{comando}, gestisce l'esecuzione ad essi corrispondente e riceve i dati di
uscita destinati all'utente. Nulla vieta che informazioni di ingresso o di
uscita siano memorizzati in file su disco.

La riga di comando ha una lunga storia ed è naturale che le siano stati
attribuiti nomi diversi per definirla nell'ambito di un particolare sistema
operativo --- \emph{linea di comando}, \emph{terminale}, \emph{console} sono fra
questi --- tuttavia i concetti che la definiscono sono ancora quelli
perfezionati da Dennis Ritchie, Ken Thompson, Brian Kernighan e da altri
programmatori esperti dei laboratori Bell sul finire degli anni '60 con Unix.

Tra le innovazioni del sistema operativo Unix, la shell rappresentava l'idea che
piccoli e veloci programmi specializzati in compiti precisi, potessero essere
concatenati in una \emph{pipeline} adottando l'output dell'elaborazione come
input per il programma successivo. La filosofia degli strumenti tende a
garantire lo sviluppo efficiente del software senza limitare la complessità
dell'elaborazione.

Oggi tra i sistemi desktop più diffusi, Mac OS X e Linux, sono basati sulla
struttura di Unix, mentre Windows ha seguito uno sviluppo parallelo replicando
ed ampliando l'ambiente di \textsc{ms-dos} con il ruolo di shell.



\section{Primi comandi}

Il formato delle istruzioni prevede in particolare la digitazione in un'unica
riga del nome del programma, seguito da eventuali opzioni e da eventuali
argomenti. L'esecuzione ha inizio premendo il tasto invio.
\medskip

\texttt{\$ \prog{nomeprogramma} \oarg{opzioni} \oarg{argomenti}}
\medskip

Solitamente le opzioni vengono distinte dai dati premettendo al loro nome uno o
due trattini \texttt{-} oppure uno slash. Se per esempio si vuol conoscere la
versione di un programma è sufficiente eseguirlo con l'opzione
\textsf{-version}. Con \textsf{pdftex}, il principale programma di composizione
del sistema \TeX{}, otteniamo:
\begin{verbatim}
$ pdftex -version
pdfTeX 3.1415926-2.5-1.40.14 (TeX Live 2013)
... eccetera
\end{verbatim}

Nel comando precedente non vi sono argomenti ma solamente un opzione che provoca
la stampa a video di un breve testo contenente le indicazioni di versione del
programma \textsf{pdftex} installato sul sistema. Altri programmi della riga di
comando utilizzano invece altre chiavi per emettere il testo d'aiuto a
dimostrazione che non esiste un unico standard rispettato da comandi o shell.

Se consideriamo l'opzione \texttt{-help} oppure \texttt{-?} si richiederà la
stampa sintetica della sintassi prevista con l'elenco delle opzioni disponibili
e brevi testi esplicativi. Proviamo al terminale:
\begin{verbatim}
$ pdftex -help
Usage: pdftex [OPTION]... [TEXNAME[.tex]] [COMMANDS]
   or: pdftex [OPTION]... \FIRST-LINE
   or: pdftex [OPTION]... &FMT ARGS
... eccetera
\end{verbatim}

Il \texttt{\$} --- o \texttt{>} per i sistemi Windows --- è chiamato
\emph{prompt} ed è il segno dopo il quale si digitano i comandi con il compito
di separare informazioni utili dai comandi stessi. Nella guida faremo uso del
\texttt{\$} riportandolo negli esempi come primo carattere a simboleggiare la
shell.

A volte i comandi possono essere interattivi ovvero, una volta avviati, possono
richiedere ulteriori dati in funzione dell'andamento dell'esecuzione.

\section{Avviare la riga di comando}\label{secAvvio}

Come vedremo, le modalità con cui si opera con la shell non sono poi così
diverse tra i vari sistemi operativi. Molti concetti sono comuni e spesso i
comandi di base hanno addirittura lo stesso nome e già questo è uno spunto di
riflessione interessante.

Solitamente la shell si presenta come una normale finestra grafica --- modalità
detta in emulazione di terminale --- al cui interno compare il cursore
lampeggiante davanti al prompt in attesa di istruzioni.

Nei prossimi paragrafi impareremo ad accedere alla shell in Windows, Mac~OS~X, e
Linux.

\subsection{La shell in Windows}

In Windows la shell è chiamata \emph{Prompt dei comandi}. Vi si accede dalla
barra dei comandi \menu{Start > Programmi > Accessori > Prompt dei comandi}.
Per comodità è possibile creare un shortcut o scorciatoia --- un piccolo file
puntatore ad un altro file --- sul Desktop o nella barra delle applicazioni,
così che basta un click per avviarla.

In Windows esiste una modalità particolare di avvio della shell che consiste
nell'aprire il dialogo \texttt{Esegui...} dal menù Start e digitare il comando
\texttt{cmd} prima di confermare su \keys{OK} o con il tasto \keys{\return}.

\subsection{La shell in Mac OS X}

In Mac la riga di comando è rappresentata dall'applicazione \textsf{Terminale},
situata nella cartella Utility all'interno della cartella Applicazioni.

Potete anche trascinare l’icona Terminale dalla directory Utility di
Applicazioni, alla \emph{Dock Bar}. Per avviare la shell fate click sull'icona
appena creata.

\subsection{I sistemi operativi Linux}

In Linux la shell è di casa. Se il vostro Desktop Environment (DE) è Gnome
utilizzerete \texttt{gnome-terminal} dal menù \menu{Applicazioni > Accessori >
Terminale}. In KDE l'emulatore di terminale è il programma \texttt{Konsole}.

Troverete molto comodo avviare il Terminale da tastiera con una combinazioni di
tasti, per esempio \keys{\ctrl + \Altwin + T}, da impostare nelle scorciatoie
del DE.

In Linux esiste anche il concetto di \emph{console virtuale}, l'insieme di 7
sessioni utente parallele, 6 con interfaccia a caratteri ed una grafica (quella
che utilizziamo normalmente). Si può passare da una all'altra premendo i tasti
funzione da \keys{F1} a \keys{F6} con la combinazione \keys{\ctrl + \Altwin +
F1-6}, per le console a caratteri ed il tasto funzione \keys{F7} nella
combinazione \keys{\ctrl + \Altwin + F7}, per la sessione grafica in cui opera
il Desktop Environment.

Le 6 console testuali sono ambienti a riga di comando di accesso indipendente al
sistema, in cui è necessario eseguire il log-in fornendo un account valido
accreditato sul sistema. Potremo così avviare fino a 7 sessioni parallele
indipendenti sullo stesso computer e con le stesse credenziali.








